\section{Responsible Research}
\label{sec:responsible}
Deep learning, among other fields, faces a reproducibility crisis \cite{semmelrock_reproducibility_2023}. We combated this issue in our research by providing a replication package, as mentioned in \Cref{sec:setup}, containing all the source code and instructions to reproduce the results presented in this paper. Furthermore, we reported our hyperparameters and hardware in \Cref{sec:setup}. All our models were trained and evaluated with fixed seeds. These techniques together ensure that our experiments and findings can be replicated exactly.

In accordance with the Netherlands Code of Conduct for Research Integrity, we reported all our experiment results in this paper under the principles of honesty and transparency. In addition, following the standards for good research practices (chapter 3), we refrained from data fabrication and manipulation. For example, our training and evaluate sets are separate and all models had access to the same data.

Our work raises ethical concerns as it is related to language models. Language models maybe be trained on copyrighted or private texts. In addition, when trained with domain knowledge, language models can be misused by malicious actors. To mitigate these issues, all our models were pretrained on a synthetic dataset. The dataset contains no knowledge of any specific domain.

% Reflect on the ethical aspects of your research and discuss the reproducibility of your methods.
% Note that although in many published works there is no such a section (it may be part of some meta-information collected by the journal, or part of the discussion section), we require you to think (and report) about this as part of this course.